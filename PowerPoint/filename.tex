%%%%%%%%%%%%%%%%%%%%%%%%%%%%%%%%%%%%%%%%%
% Beamer Presentation
% LaTeX Template
% Version 1.0 (10/11/12)
%
% This template has been downloaded from:
% http://www.LaTeXTemplates.com
%
% License:
% CC BY-NC-SA 3.0 (http://creativecommons.org/licenses/by-nc-sa/3.0/)
%
%%%%%%%%%%%%%%%%%%%%%%%%%%%%%%%%%%%%%%%%%

%----------------------------------------------------------------------------------------
%	PACKAGES AND THEMES
%----------------------------------------------------------------------------------------

\documentclass{beamer}

\mode<presentation> {

% The Beamer class comes with a number of default slide themes
% which change the colors and layouts of slides. Below this is a list
% of all the themes, uncomment each in turn to see what they look like.

%\usetheme{default}
%\usetheme{AnnArbor}
\usetheme{Antibes}
%\usetheme{Bergen}
%\usetheme{Berkeley}
%\usetheme{Berlin}
%\usetheme{Boadilla}
%\usetheme{CambridgeUS}
%\usetheme{Copenhagen}
%\usetheme{Darmstadt}
%\usetheme{Dresden}
%\usetheme{Frankfurt}
%\usetheme{Goettingen}
%\usetheme{Hannover}
%\usetheme{Ilmenau}
%\usetheme{JuanLesPins}
%\usetheme{Luebeck}
%\usetheme{Madrid}
%\usetheme{Malmoe}
%\usetheme{Marburg}
%\usetheme{Montpellier}
%\usetheme{PaloAlto}
%\usetheme{Pittsburgh}
%\usetheme{Rochester}
%\usetheme{Singapore}
%\usetheme{Szeged}
%\usetheme{Warsaw}

% As well as themes, the Beamer class has a number of color themes
% for any slide theme. Uncomment each of these in turn to see how it
% changes the colors of your current slide theme.

%\usecolortheme{albatross}
%\usecolortheme{beaver}
%\usecolortheme{beetle}
%\usecolortheme{crane}
%\usecolortheme{dolphin}
%\usecolortheme{dove}
%\usecolortheme{fly}
%\usecolortheme{lily}
%\usecolortheme{orchid}
%\usecolortheme{rose}
%\usecolortheme{seagull}
%\usecolortheme{seahorse}
%\usecolortheme{whale}
%\usecolortheme{wolverine}

%\setbeamertemplate{footline} % To remove the footer line in all slides uncomment this line
%\setbeamertemplate{footline}[page number] % To replace the footer line in all slides with a simple slide count uncomment this line

\setbeamertemplate{navigation symbols}{} % To remove the navigation symbols from the bottom of all slides uncomment this line
}

\usepackage{newtxtext,newtxmath}
%\usepackage{filecontents}
\usepackage[backend=biber,bibstyle=numeric-comp]{biblatex}
%\addbibresource{sample.bib}
\usepackage{graphicx} % Allows including images
\usefonttheme{serif}
\setbeamertemplate{footline}[frame number]
\usepackage{booktabs} % Allows the use of \toprule, \midrule and
\usepackage{listings}
\usepackage{fancyhdr}
%\bottomrule in tables
\usepackage{etoolbox}
\usepackage{hyperref}

% \let\origtau\tau % save the original form of '\tau'
% \renewcommand{\tau}{\scalebox{1.44}{$\origtau$}}

%----------------------------------------------------------------------------------------
%	TITLE PAGE
%----------------------------------------------------------------------------------------

\title[Paper]{\href{https://arxiv.org/abs/2005.11401}{Paper Title}} % The short title appears at the bottom of every slide, the full title is only on the title page
\subtitle{Conference name}
\author{Author name} % Your name
\institute[National Yang Ming Chiao Tung University, Hsinchu] % Your institution as it will appear on the bottom of every slide, may be shorthand to save space
{
Speaker: Your name
}
\date{May 30, 2023} % Date, can be changed to a custom date
\begin{document}

\begin{frame}
\titlepage % Print the title page as the first slide
\includegraphics[width=0.2\textwidth]{logo/en_nycu.png}
\end{frame}

\begin{frame}
\frametitle{Table of contents} % Table of contents slide, comment this block out to remove it
\tableofcontents[hideallsubsections] % Throughout your presentation, if you choose to use \section{} and \subsection{} commands, these will automatically be printed on this slide as an overview of your presentation
\end{frame}

%----------------------------------------------------------------------------------------
%	PRESENTATION SLIDES
%----------------------------------------------------------------------------------------

\begin{frame}
\frametitle{Abstract}

\end{frame}

%------------------------------------------------

\section{Introduction} % Sections can be created in order to organize your presentation into discrete blocks, all sections and subsections are automatically printed in the table of contents as an overview of the talk

\begin{frame}
\frametitle{Table of contents} % Table of contents slide, comment this block out to remove it
\tableofcontents[	currentsection,
	currentsubsection, 
	hideothersubsections, 
	sectionstyle=show/shaded, ] % Throughout your presentation, if you choose to use \section{} and \subsection{} commands, these will automatically be printed on this slide as an overview of your presentation
\end{frame}

%------------------------------------------------

\begin{frame}
\frametitle{Introduction}

\end{frame}

%------------------------------------------------

\section{Methods} % Sections can be created in order to organize your presentation into discrete blocks, all sections and subsections are automatically printed in the table of contents as an overview of the talk

\begin{frame}
\frametitle{Table of contents} % Table of contents slide, comment this block out to remove it
\tableofcontents[	currentsection,
	currentsubsection, 
	hideothersubsections, 
	sectionstyle=show/shaded, ] % Throughout your presentation, if you choose to use \section{} and \subsection{} commands, these will automatically be printed on this slide as an overview of your presentation
\end{frame}

%------------------------------------------------

\begin{frame}
\frametitle{Methods}

\end{frame}

%------------------------------------------------
\subsection{Models}
\begin{frame}
\frametitle{Models}

\end{frame}

%------------------------------------------------
\subsection{Training}
\begin{frame}
\frametitle{Training}

\end{frame}

%------------------------------------------------

\section{Experiments} % Sections can be created in order to organize your presentation into discrete blocks, all sections and subsections are automatically printed in the table of contents as an overview of the talk

\begin{frame}
\frametitle{Table of contents} % Table of contents slide, comment this block out to remove it
\tableofcontents[	currentsection,
	currentsubsection, 
	hideothersubsections, 
	sectionstyle=show/shaded, ] % Throughout your presentation, if you choose to use \section{} and \subsection{} commands, these will automatically be printed on this slide as an overview of your presentation
\end{frame}

%------------------------------------------------

\begin{frame}
\frametitle{Experiments}

\end{frame}

%------------------------------------------------

\section{Results} % Sections can be created in order to organize your presentation into discrete blocks, all sections and subsections are automatically printed in the table of contents as an overview of the talk

\begin{frame}
\frametitle{Table of contents} % Table of contents slide, comment this block out to remove it
\tableofcontents[	currentsection,
	currentsubsection, 
	hideothersubsections, 
	sectionstyle=show/shaded, ] % Throughout your presentation, if you choose to use \section{} and \subsection{} commands, these will automatically be printed on this slide as an overview of your presentation
\end{frame}

%------------------------------------------------

\section{Related Work} % Sections can be created in order to organize your presentation into discrete blocks, all sections and subsections are automatically printed in the table of contents as an overview of the talk

\begin{frame}
\frametitle{Table of contents} % Table of contents slide, comment this block out to remove it
\tableofcontents[	currentsection,
	currentsubsection, 
	hideothersubsections, 
	sectionstyle=show/shaded, ] % Throughout your presentation, if you choose to use \section{} and \subsection{} commands, these will automatically be printed on this slide as an overview of your presentation
\end{frame}

%------------------------------------------------

\begin{frame}
\frametitle{Related Work}

\end{frame}

%------------------------------------------------

\section{Discussion} % Sections can be created in order to organize your presentation into discrete blocks, all sections and subsections are automatically printed in the table of contents as an overview of the talk

\begin{frame}
\frametitle{Table of contents} % Table of contents slide, comment this block out to remove it
\tableofcontents[	currentsection,
	currentsubsection, 
	hideothersubsections, 
	sectionstyle=show/shaded, ] % Throughout your presentation, if you choose to use \section{} and \subsection{} commands, these will automatically be printed on this slide as an overview of your presentation
\end{frame}

%------------------------------------------------

\begin{frame}
\frametitle{Discussion}

\end{frame}

%------------------------------------------------
\end{document}